\chapter{BITCOIN AND THE BLOCKCHAIN [R]EVOLUTION}
\label{ch:Bitcoin}

We stand at the verge of a technological revolution that will fundamentally change the way we live our lives, perhaps even without us being aware of all of it. The immense scale, scope and complexity of this transformation will be unlike anything we have experienced before, and we cannot yet be confident how this scenario is going to play out. One could argue this technological revolution will have the same impact as the internet had if not more.\medskip


    \tcbset{colback=orange!3!white,fonttitle=\bfseries}
\begin{tcolorbox}
[enhanced,
title=Internet is everywhere,
frame style=
{left color=orange!85!black,right color=yellow!95!black}]

Think of all the applications that make use of the internet (what does not?). Think about the daily lives of people around the world and about the apps we all use daily. For example, think about the impact e-mail had historically. First, it took at least five days to send a letter from Switzerland to South Africa, with the invention of e-mail this duration was instantly reduced to a mere couple of seconds. The services provided by the internet have propelled forward its rates of adoption.\footnote{\emph{Watch the internet as it grows in real-time.} [20-10-2018] \url{http://www.internetlivestats.com}}

\end{tcolorbox}
\medskip


Blockchain and distributed ledger technology will have a dramatic impact on business and society, by providing a secure, direct way of exchanging money, intellectual property and other rights and assets without the involvement of traditional intermediaries like banks, utility companies and governments. 

 \medskip
    \tcbset{colback=orange!3!white,fonttitle=\bfseries}
    \begin{tcolorbox}
    [enhanced,
    title=Digitization 2.0: The Internet of Value,
    frame style=
    {left color=orange!85!black,right color=yellow!95!black}]
    
\textit{\say{Blockchain and distributed ledger technology might represent a second era of the internet or the Digital Age. For the last 40 years, we've had the internet of information; now, with blockchain and distributed ledger technology, we're getting the internet of value.}}
\end{tcolorbox}
\medskip

\section{Distributed Ledger Technology}


Blockchain is the ingeniously simple, revolutionary protocol that allows transactions to be simultaneously anonymous and secure by maintaining a tamper-proof public ledger of value. Though its the technology that drives Bitcoin and other cryptocurrencies, the underlying framework has the potential to go far beyond. It has real utility and can record virtually everything of value to humankind, from birth and death certificates to land keeping records and even votes. 
Blockchain is a specific type of distributed ledger. It is designed to record transactions or digital interactions and bring much-needed transparency, efficiency, and added security to businesses. But these two technologies are not the same; blockchain is just the tip of the proverbial iceberg.

\begin{quotation}

  \textit{\say{The technology likely to have the greatest impact on the future of the world economy has arrived, and it is not self-driving cars, solar energy or artificial intelligence. Its called the blockchain and it is here to stay.}}
  \begin{flushright}
    \small{--- \textbf{Tapscott, Don}}
  \end{flushright}

\end{quotation}

\section{A Paradigm Shift}
The fact that the blockchain and distributed ledger technology is causing a paradigm shift in the financial services industry is undeniable. Where governments and financial institutions now have the power, peer-to-peer distributed cryptocurrencies such as Bitcoin provide us with a potential tool to escape the fiat currency system and enables peer-to-peer exchange of value. 

Blockchain poses a threat to governments and financial institutions whose business models are based on the roles of intermediaries and their power to control and influence people's behaviour. It is hard to imagine that banks and financial institutions might disappear entirely simply because they have much influence and are usually labelled as \say{too big to fail}. More likely is the fact that traditional legacy infrastructure will have to keep up with these innovations and will adopt their versions of it. There will be centralised and decentralised systems operating next to each other. Both with their pros and cons. Most likely, it will be a combination of both, until decentralised systems and networks are capable of completely replacing our current centralised infrastructure without losing any of its core capabilities.\medskip


\section{Legacy financial infrastructures}
Blockchain poses a threat to governments and financial institutions whose business models are based on the roles of intermediaries and their power to control and influence people's behaviour. It is hard to imagine that banks and financial institutions might disappear entirely simply because they have much influence and are usually labelled as \say{too big to fail}. More likely is the fact that traditional legacy infrastructure will have to keep up with these innovations and will adopt their versions of it. There will be centralised and decentralised systems operating next to each other. Both with their pros and cons. Most likely, it will be a combination of both, until decentralised systems and networks are capable of completely replacing our current centralised infrastructure without losing any of its core capabilities.

\section{Bitcoin}
Bitcoin started it all, genuine person to person (peer-to-peer) digital currencies that do not belong to, are not controlled by, any state or any corporation.  What is unique about Bitcoin is that it represents the open blockchain, the blockchain that offers censorship resistance, open access and innovation without permission. This engine creates an explosion of innovative potential. 
\medskip


 \begin{quotation}

      \textit{\say{Bitcoin is the first borderless, transnational, open system of access for financial payments and trust that enables innovation without permission, with high resistance to censorship, coercion and political manipulation. Bitcoin is a mathematical proof system that is fundamentally neutral to participants. It exhibits a principle that on the internet, we call net neutrality and brings that to finance. Which of course terrifies some groups of people, because what about authority?}}
      \begin{flushright}
        \small{--- \textbf{Antonopoulos, Andreas}}
      \end{flushright}
    
\end{quotation}


\medskip

    \tcbset{colback=orange!3!white,fonttitle=\bfseries}
    \begin{tcolorbox}
    [enhanced,
    title=The power of open blockchain,
    frame style=
    {left color=orange!85!black,right color=yellow!95!black}]
            
        \textit{\say{What the internet did for communication, open blockchain is doing for finance. It is introducing a fundamentally different, network-centric, and flat system that allows us to perform transactions without the permission of authority and intermediaries. Bitcoin derives trust from the collaboration and computation of thousands of nodes and it uses a blockchain. The open blockchain and other open source projects will change the world. We cannot predict that this will be the only open blockchain that matters, nor that it will be the dominant one. The first car that we invented is not the car we have today. Why? Because it was not the best car, and it was the beginning of the automobile industry, which took time to develop the technology further. However, with technology, the best technology does not always necessarily win. So far, it is the only open blockchain, that operates on the public internet, that is tested on such a big scale, 24 hours a day, seven days a week, and it survives. Moreover, when Bitcoin survives, it is getting stronger every day. \cite{future_of_money}}}
        \end{tcolorbox}
        \medskip


