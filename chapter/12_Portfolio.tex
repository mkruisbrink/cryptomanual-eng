\chapter{PORTFOLIO MANAGEMENT}

Portfolio management is all about making decisions. With markets, you can identify strengths, weaknesses, threats, and opportunities. While working toward a balanced investment portfolio, you are continuously aware of the context of your investment by reading up on developments and performing research. Please note that your investment portfolio is a reflection of your decisions, and if you make bold or ill-calculated moves, you will potentially end up with high risk throughout your investment portfolio. There are several essential concepts of which you should take note. Let us first discuss the critical portfolio management concepts and continue with the main types of investments and asset classes.\medskip 

 \medskip
    \tcbset{colback=orange!3!white,fonttitle=\bfseries}
    \begin{tcolorbox}
    [enhanced,
    title=Opportunity Management,
    frame style=
    {left color=orange!85!black,right color=yellow!95!black}]
        
            \textit{Some investment opportunities are considered \say{high risk, potentially high reward}. Think about the profitability of investing in internet applications and protocols back in the day. Imagine investing in companies such as Apple and Facebook when they were still operating out of their garage. Think of Bitcoin and the development thereof. In 2008, many people lost trust in the banking system, invested in Bitcoin and took some serious risks. In hindsight, this has been a truly golden and almost once in a lifetime opportunity. What if we tell you there is more?}
       
    \end{tcolorbox}


\section*{Asset Classes - Types of Investments}

An asset class is a group of similar investments based on having similar characteristics and financial structure. These investments are typically traded in the same financial markets, subject to the same rules and regulations, and you can categorize them by location. Investors and market analysts often view investments in domestic securities, foreign or international investments, and investments in emerging markets as different categories of investments. 

\section{Stocks or Equities}
Equities are shares of ownership that are issued by publicly-traded companies and traded on stock exchanges (such as the New York Stock Exchange, NASDAQ, London Stock Exchange, and Hong Kong Stock Exchange) You can potentially profit from equities either through a rise in the share price or by receiving dividends.

\section{Real Estate and Land}
Real estate is a real (tangible) property made up of land as well as anything on it, including buildings, flora and fauna and natural resources such as crops, minerals or water. Real estate investing involves the purchase, ownership, management, rental or sale of real estate for profit. Real-estate is an asset form with limited liquidity relative to other investments; it is also capital intensive and is highly cash flow dependent. If these factors are not well understood and managed by the investor, real estate could become a risky investment.

\section{Commodities}
Commodities are basic goods used in commerce which are interchangeable with other commodities of the same type.
Some of the more commonly traded commodities are gold, beef, lumber, oil and natural gas. Other examples include silver, copper, iron ore, salt, sugar, tea, coffee and more - you get the picture.
Commodities are used as inputs in the production of other goods or services. The quality of a given commodity may differ slightly, but it is essentially uniform across producers. When participants buy or sell commodities on an exchange, they must also meet specified minimum standards, also known as a basis grade.

\section{Bonds and Fixed Income Investments}
Bonds and other fixed-income investments are investments in debt securities such as corporate or government bonds that pay a fixed rate of return in the form of interest. 

\section{Futures and other Financial Derivatives}
This category includes futures contracts, the foreign exchange (forex, FX) market, options, and an expanding array of financial derivatives, i.e., financial instruments that are based on or derived from, an underlying asset. For example, stock options are a derivative of the underlying stocks.

\section{Cash or Cash Equivalents}
The primary advantage of cash or cash equivalent investments is their liquidity. Money held in the form of cash or cash equivalents can be quickly and easily accessed and traded at any time. 

\section{Cryptocurrencies}
ICOs and cryptocurrencies (such as Bitcoin) can also be considered to be a subclass of cash or cash equivalents but although they are less liquid, more volatile and also often considered as high risk. Cryptocurrencies have high potential, and the long bear-market of 2018 is again providing plenty of opportunity for the avid risk-taker.

\subsection{Cryptocurrencies - categories and classification}
The market for cryptocurrencies and tokens has exploded. The cryptocurrency space got its start with Bitcoin, a decentralized peer-to-peer cryptocurrency. Then Ethereum created a functional token, Ether (ETH) that acts as \say{gas} to a decentralized computer
operating system. Many other tokens followed. A dynamic, rapidly-growing market to buy, sell and trade cryptocurrencies and tokens has been the result.
There are now thousands of tokens - also, a myriad of token offerings, often known as a \say{ICO}. While there is substantial public interest and demand for cryptocurrencies and tokens, there is little knowledge and structure to classify and organize tokens. The names that label tokens are mostly organic and not yet standardized\footnote{Ethos (2017/2018); \href{https://www.ethos.io/blockchain-finance/}{Ethos Token Classification System: A Framework for Understanding Types of Blockchain Based Tokens}.}. 



 \medskip
    \tcbset{colback=orange!3!white,fonttitle=\bfseries}
    \begin{tcolorbox}
    [enhanced,
    title=Class 1: Utility and Functional Tokens,
    frame style=
    {left color=orange!85!black,right color=yellow!95!black}]

            \say{The first class of tokens is functional or utility tokens. The values of these tokens are derived from the thing to which they provide access. A bridge token that allows access to a bridge has a price derived by how much people are willing to pay to cross the bridge. Ether has a price that is derived by how much people are willing to pay for access to the Ethereum Virtual Machine or the distributed worldwide computer that runs Ethereum Applications. ETHOS is an ERC20 functional token spawn out of the Ethereum ecosystem that builds on the Ethereum Protocol Layer 4 as a set of applications providing access to the Ethos ecosystem.
            This ecosystem allows users, businesses, and institutions the ability to create financial applications powered by open standards. These tokens are closest to virtual commodities like oil, gas, steel, grain or gold.}
       
    \end{tcolorbox}



 \medskip
    \tcbset{colback=orange!3!white,fonttitle=\bfseries}
    \begin{tcolorbox}
    [enhanced,
    title=Class 2: Cryptocurrencies or Transactional Tokens,
    frame style=
    {left color=orange!85!black,right color=yellow!95!black}]

        
            \say{The second class of tokens is transactional tokens or cryptocurrencies. These tokens don't have any \say{inherent} value, but they derive value from the network effect and belief that these tokens have value.
            Similarly, fiat currency is widely accepted as something that has value; Bitcoin has been widely accepted by people around the world to have value. Cryptocurrencies are harder to understand intuitively and are often criticized as being bubbles with no fundamental value backing them. These tokens are closest to the virtual property since they are scarce, but unlike property, you can't use them as a place to live. It is unclear whether currencies in this category should be regulated as property, commodities or even currencies.}
       
    \end{tcolorbox}

 \medskip
    \tcbset{colback=orange!3!white,fonttitle=\bfseries}
    \begin{tcolorbox}
    [enhanced,
    title=Class 3: Representation - Tokenized Securities,
    frame style=
    {left color=orange!85!black,right color=yellow!95!black}]
            \say{The third type of token is tokenized securities. These tokens derive value by representing something else of value - the most common being a share of a company. The value proposition behind these tokens is evident, but need to be highly regulated just like any securities market and exchange should be.
            Ethereum provides an incredibly powerful platform for tokenized securities and the means to regulate them if appropriately constructed. Tokens that fall under this category should be regulated as securities.} 
       
    \end{tcolorbox}


\section{Other Investments}
These might include artwork, various other collectibles, peer to peer lending, hedge funds, private investments (private equity, venture capital, angel investing). 
Generally speaking, alternative investments experience less liquidity and are riskier. Risk and liquidity are both depending on the position in the business cycle. 



\section{Key Concepts}
All investors should understand a few essential investment concepts, including how to evaluate investment performance, asset allocation, diversification, rebalancing and the role that risk plays in virtually all aspects of investing.\footnote{Finra; \href{https://www.finra.org/investors/key-investing-concepts}{Key Investing Topics}.}


\subsection{Evaluation of investment performance}
Choosing investments is just the beginning of your work as an investor. As time goes by, you'll need to monitor the performance of these investments to see how they are working together in your portfolio to help you progress toward your goals. Generally speaking, progress means that your portfolio value is steadily increasing, even though one or more of your investments may have lost value.

If your investments are not showing any gains or your account value is slipping, you'll have to determine why and decide on your next move. Also, because investment markets change all the time, you'll want to be alert to opportunities to improve your portfolio's performance, perhaps by diversifying into a different sector of the economy or allocating part of your portfolio to international investments. To free up money to make these new purchases, you may want to sell individual investments that have not performed well, while not abandoning the asset allocation you've selected as appropriate.


\subsection{Asset allocation}
When you allocate your assets, you decide - usually on a percentage basis - what portion of your total portfolio to invest in different asset classes, like stocks, bonds, and cash or cash equivalents. You can make these investments either directly by purchasing individual securities or indirectly by choosing funds that invest in those securities. As you build a more extensive portfolio, you may also include other asset classes, such as real estate, which can help spread out, and thus moderate, your investment risk.

Asset allocation is a useful tool to manage systematic risk because different categories of investments respond to changing economic and political conditions in different ways. By including different asset classes in your portfolio, you increase the probability of some of your investments providing satisfactory returns even if others are flat or losing value. The practice of reducing investment portfolio risk by diversifying your investments across different asset classes and subclasses is referred to as asset allocation. An example portfolio and some generic portfolio allocations are shown in \cref{fig:portfolio example,fig:portfolio allocation,fig:subclasses} - let your imagination do the rest.

\begin{figure}
    \centering
   
    \begin{tikzpicture}
        
        \pie [radius=3, text=legend]{20/precious metals, 20/cryptocurrencies, 20/cash equivalents, 10/stocks or equities, 10/land or real-estate, 10/investment funds, 10/alternative investments}
   
    \end{tikzpicture}
    
    \caption{Portfolio allocation example}
    \label{fig:portfolio example}
\end{figure}

\begin{figure}
    \centering
   
    \begin{tikzpicture}
        \pie [pos={0,0}, radius=2]{80/,20/}
        \pie [pos={4.5,0}, radius=2]{40/,30/,20/,10/}
        \pie [pos={9,0}, radius=2]{30/,25/,15/,10/,10/,5/,5/}
    \end{tikzpicture}
    
    \caption{Generic portfolio allocation - left to right, high to lower risk}
    \label{fig:portfolio allocation}
\end{figure}

\begin{figure}
    \centering
\begin{tikzpicture}
    \pie [pos={0,0}, radius=2, text=pin, rotate=145, explode={0,0,0.2,0.2}]{10/A, 20/B, 30/C, 40/D}
    \pie [pos={9,0}, radius=2, rotate=90, color={yellow!60, yellow!40}, explode =0] {60/C1, 40/C2}
    \pie [pos={4.5,0}, radius=2, rotate=180, color={orange!40, orange!30, orange!30}, explode =0] {40/D1, 30/D3, 30/D2}
\end{tikzpicture}
    \caption{Generic representation of sub-classes within asset classes}
    \label{fig:subclasses}
\end{figure}


\subsection{Diversifying your portfolio}
When you diversify, you aim to manage your risk by spreading out your investments. You can expand and diversify both within and among different asset classes. You can also diversify within asset classes, in which case they are referred to as sub-classes. For example, you divide the money you've allocated to a particular asset class, such as stocks, among various categories of investments that belong to that asset class. An investor may own different commodities such as gold, silver and palladium, but at the same time, trade in oil and coffee beans. By simultaneously holding multiple sub-classes that are not correlated, you are effectively reducing risk. 
\begin{quotation}

  \textit{\say{It often requires some (catastrophic) crisis to see what is in plain sight. When the next financial crisis hits us, the world economy, especially the financial sector and currencies, will suffer a massive blow, and billions of people will not have seen it coming. It is time to prepare - diversify out of fiat currencies, because this time it is global.}}
  \begin{flushright}
    \small{--- \textbf{Cryptomanuals}}
  \end{flushright}

\end{quotation}

\subsection{Re-balancing your portfolio}
As market performance alters the values of your asset classes, you may find that your asset allocation no longer provides the balance of growth and return that you want. In that case, you may want to consider adjusting your holdings and stabilizing your portfolio. Assets grow at different rates - which means that your portfolio might end up out of line with the allocation you have chosen. For example, some assets might recently have grown at a much faster rate. To compensate, you might reallocate some of the value of fast-growing assets into assets with slower recent growth, which may now be poised to pick up steam while recent high-performers slow down. Otherwise, you might end up with a portfolio that carries more risk and provides a smaller long-term return than you intended.\medskip 

You can reallocate holdings in your portfolio in different ways to bring it back in line with the allocation balance you intend it to have. Here are three common approaches to rebalancing:

\begin{enumerate}[label=(\alph*)]
    \item Redirect money to the lagging asset classes until they return to the percentage of your total portfolio that they held in your original allocation.
    \item Add new investments to the lagging asset classes, concentrating a larger percentage of your contributions to those classes.
    \item Sell off a portion of your holdings within the asset classes that are outperforming others. You may then reinvest the profits in the lagging asset classes.
\end{enumerate}

How you rebalance your portfolio depends entirely on your preferences and the current situation. In general, people are more comfortable with the first two alternatives than the third. People often find themselves attached to investments, which are performing well and find it hard to let them go. If they let them go, they would be able to take profits and invest these profits into the under-performing investments. 

Moreover, remember that if you invest in the lagging and under-performing classes, you will be positioned to benefit quite handsomely if they turn around and begin to thrive again. Don't become too emotionally attached to the investments, or you won't be ready to sell them when the opportunity arises.

\newpage

\section{Recommendations}
There are some essentials, which require additional attention when investing or trading and managing your investment or trading portfolio. It is not the most exhaustive list but includes that which we have found to be crucial so far, and we will continue to modify and fine-tune the list as we progress and develop. We want to emphasize that many things ultimately come down to your preference, personality and style - the way you handle it. This information serves as an anchor for people with limited knowledge and experience regarding these practices. Ultimately, it comes down to the investors doing their research. Basic fundamental analysis is considered an absolute minimum requirement for any investment. Pair this with some technical analysis for the optimal entry point, and you will be off to a good start.

\begin{enumerate}[label=\Roman*]
    \setlength\itemsep{0em}
     \item Research. Especially regarding cryptocurrencies, there are more projects than you can count and plenty of use cases in almost every industry niche. There's an astonishing amount of investment opportunity in the cryptocurrency sector, but it is extremely speculative unless you are well aware of the landscape and know what you are doing. Rash decisions can easily make you lose your investments.
    \item Never invest more than you can afford to lose. Invest using expendable income only; you should never be using lines of credit or rent money.
    \item Patience. Don't go on a shopping spree if investing for the long term and acquire too many (undiversified) assets too soon. Think in years, months and quarters instead of months, weeks and days and always have some reserves for when prices are low.
    \item Asset allocation. Don't spread your positions too thin over too many assets (i.e. twenty different assets with small positions). Again, diversification is key.
    \item Diversification. Try to cover different industries and market niches (cryptocurrencies, commodities - precious metals such as gold or silver, stocks or equities, real estate or land).
    \item Rebalancing and re-evaluating your portfolio occasionally is a perfect exercise. Make deliberate choices based on real and verified information.
\end{enumerate}
