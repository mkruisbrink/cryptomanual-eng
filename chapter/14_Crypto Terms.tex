\chapter{Crypto World}
\label{ch:Crypto Terms}
Answers to some of the most frequently asked questions and definitions of terminology used in the crypthosphere. In the second part you will find some of the crypto slang used in the crypto community.
\medskip
\section{Crypto Definitions}



\paragraph{What is Bitcoin (BTC)} Bitcoin is the world's first decentralized peer-to-peer cryptocurrency. Released in 2009 by the anonymous Satoshi Nakamoto, Bitcoin is the most significant application of blockchain technology. In its documentation, Satoshi Nakamoto laid out its implementation: a secure, global, immutable public ledger as a record of every transaction on the network (the blockchain) and an incentive for consumers to contribute their computing power to the blockchain network (Bitcoin).

\paragraph{How does Bitcoin work?} To understand how Bitcoin works, we must first understand what is a network. A network is fundamentally a system with multiple nodes (users) and connections (transactions) between these nodes. As a network, the Bitcoin Network’s purpose is to enable users to send tokens to one another (Bitcoins). As a network, the Bitcoin Network's purpose is to enable users to send tokens to one another (Bitcoins). The primary issues with transactions in general are those of security and accountability, both of which require solutions to prevent fraud. Imagine making a financial transaction over the internet in which you send your rent to your landlord. You would hope for two things: that the money does not get intercepted by a third party along the way, and that your transaction is recorded in a ledger somewhere should the payment not go through. The blockchain's approach to solving these issues is what makes it a truly revolutionary technological innovation. 

\paragraph{What is Blockchain?} The blockchain is a breakthrough technology that powers many of the cryptocurrency networks that we recognize today. At a general level, blockchain technology showcases some of the most groundbreaking solutions to networking issues in human history. The decentralized control of each cryptocurrency works through distributed ledger technology, typically a blockchain. The act of embedding a previous block of data into the current block of data is called chaining, hence, the name blockchain.

\paragraph{What is Distributed Ledger Technology?} One form of distributed ledger design is the blockchain system, which can be either public or private. A distributed ledger (also called a shared ledger or distributed ledger technology or DLT) is a consensus of replicated, shared, and synchronized digital data geographically spread across multiple sites, countries, or institutions. There is no central administrator or centralized data storage. A peer-to-peer network is required as well as consensus algorithms to ensure replication across nodes is undertaken. 

\paragraph{What is Proof of Work (PoW)?} Proof of work refers to the computational puzzle that miners have to solve which allows many open blockchain networks to remain secure and decentralized. Proof of Work uses cryptographic functions that essentially guarantee a certain number of computer cycles were spent to solve the puzzle. In other words, by solving this puzzle, you are proving that you did some amount of work - hence the term Proof of Work. 

\paragraph{What are so-called Miners?} Bitcoin requires the help of miners: individuals or organizations that contribute their computational energy to encrypt the information of these transactions. Miners will use large rigs of hardware to solve cryptographic puzzles on a Bitcoin-specific computer program that hashes the information of each transaction on the network. When a transaction is encrypted by a miner, it is added as a new block to a connected chain of blocks. This acts as a visible, immutable public ledger - hence the name blockchain. In exchange for all of this computational effort, if the miner's creation of the new block is deemed valid by the network (via consensus), the miner is rewarded with a newly created Bitcoin. The miner can then use this coin however he/she chooses.

\paragraph{What is Ethereum (ETH)?} Ethereum is an open platform that enables developers to build and deploy decentralized applications such as smart contracts and other complex legal and financial applications. You can think of Ethereum as a programmable Bitcoin where developers can use the underlying blockchain to create markets, shared ledgers, digital organizations, and other endless possibilities that need immutable data and agreements, all without the need for a middleman.

\paragraph{What are Smart Contracts?} 
A smart contract is an application that runs on a blockchain network.These self-executing, digital transactions are known as smart contracts; a contract between two parties that executes itself as an automated digital transaction once it has been signed. Deployed on public blockchain networks, smart contracts are self-executing and immutable after their signing. The uses of such contracts are limitless, as they can be used to build decentralized exchanges, tokenized assets, games and more. Since the first smart contract platform, Ethereum, was released in 2015, it has become the primary focus of innovators in the blockchain arena.

\paragraph{What is Proof of Stake (PoS)?} The Proof of Stake (PoS) was created as an alternative to the Proof of Work (PoW), to tackle inherent issues like energy usage. Proof of Stake concept states that a person can mine or validate block transactions according to how many coins he or she holds. The proof of stake (PoS) seeks to address this issue by attributing mining power to the proportion of coins held by a miner. This way, instead of utilizing energy to answer PoW puzzles, a PoS miner is limited to mining a percentage of transactions that is reflective of his or her ownership stake. For instance, a miner who owns 3\% of the Bitcoin available can theoretically mine only 3\% of the blocks.

\paragraph{What is a cryptocurrency?} A cryptocurrency is a digital asset designed to work as a medium of exchange that uses strong cryptography to secure and record (financial) transactions, control the creation of additional units, and verify the validity and transfer of assets. Cryptocurrencies are typically deployed on a decentralized network as opposed to centralized digital currency and central banking systems.

\paragraph{What is a Peer to Peer Network?} P2P Stands for "Peer to Peer." In a P2P network, the "peers" are computer systems which are connected to each other via the Internet. Peer to peer networks are usually formed by groups of computers. These computers all store their data using individual security but also share data with all the other nodes. The nodes in peer to peer networks both use resources and provide resources. Files can be shared directly between systems on the network without the need of a central server. 

\paragraph{What is a Peer to Peer service?} A peer-to-peer (P2P) service is a decentralized platform whereby two individuals interact directly with each other, without intermediation by a third-party. Instead, the buyer and the seller transact directly with each other via the P2P service. 

\paragraph{How do cryptocurrencies work?}
The decentralized control of each cryptocurrency works through distributed ledger technology, typically a blockchain, that serves as a public financial transaction database.

\paragraph{The first decentralized cryptocurrency}
Bitcoin, first released as open-source software in 2009, is generally considered the first decentralized cryptocurrency. Since the release of bitcoin, over 6,000 altcoins (alternative variants of bitcoin, or other cryptocurrencies) have been created.


\paragraph{How do cryptocurrency wallets work?}
Unlike traditional [pocket] wallets, digital wallets do not store any currency. Cryptocurrencies do not get stored in any single location or exist anywhere in any physical form. All there is are records of transactions stored on the blockchain.

\paragraph{What are cryptocurrency wallets?}
Cryptocurrency wallets are software programs that work with public and private key pairs. The wallets provide a user interface via which you can monitor their recorded balance on the network and conduct transactions. When a person sends you Bitcoin or any other type of digital currency, they are permanently signing off ownership of the coins linked to their wallet address, and once the transaction is verified, the network records Bitcoins ownership on your address.

\paragraph{Recovery or Seed phrases}
Anytime you set up a wallet; users are provided with a unique recovery seed composed of anywhere from 12 to 24 randomized words (12 and 24 being most common). You are encouraged to write this recovery seed down somewhere safe and never to post it online. Recovery seeds are considered the most crucial aspect of maintaining the safety of your cryptocurrencies. A recovery seed is your best friend when you lose your paper, hardware, or mobile wallet, as its the only way you can recover your funds and wallet. In summary, if you ever lose access to your wallet for whatever reasons, the seed phrase is what you require to recover your `lost` funds. 

\paragraph{Cryptography}
The field of cryptography is fundamental to many cryptocurrencies such as Bitcoin. Cryptography is the practice of secure communication in the presence of third parties. In other words, cryptography allows for data to be stored and communicated in such a way that third parties are prevented from reading the contents. Cryptography is utilized in the creation of public and private keys to make cryptocurrency systems a secure network upon which users can safely operate.

\paragraph{Ownership on a cryptocurrency}

The concept of ownership on a cryptocurrency system is primarily comprised of three interconnected elements:
\begin{list}
\item{Public and private keys}
\item{Public addresses (hashed public key)}
\item{Digital signatures}
\end{list}


\paragraph{What are public and private keys?}
When dealing with cryptocurrency, users are given a public address and a private key to send and receive coins or tokens. The public address is where the funds are deposited and received. However, even though a user has tokens deposited into his address, he won\'t be able to withdraw them without the unique private key. The public address is generated by hashing the public key. In turn, the public key is created from the private key through a complicated mathematical algorithm (hashing). However, it is near impossible to reverse the process by generating a private key from a public key. 


\paragraph{What is a cryptocurrency public address}
Your public address is essentially a secured cryptographic representation of your public key. Think of this address as having a mailbox with a specific string of numbers attached as an identifier and only you can access it with your private key. Some examples of different cryptocurrency public keys or addresses are listed in \cref{tab:publickeys}. Usually, these public addresses will have an automated copy button available and will also be shown as a QR code so you can easily scan it with a mobile device. Remember that for each public address you have both a public and a private key that match it!


\paragraph{Digital Signatures}
Digital signatures play an important role in cryptocurrency systems because they prove ownership of funds and allow the individual in control of those funds to spend them. With Bitcoin, a digital signature is effectively intended to serve three distinct purposes:
\item{A digital signature serves as proof that the owner of a private key, who will by extension have ownership of his/her funds, has indeed authorized that those funds can be spent.}
\item{A digital signature serves as proof that the authorization is undeniable.}
\item{digital signature proves that the transaction that has been authorized by the signature has not or cannot be modified by anyone after it has been signed.}
 
 
\paragraph{Transactions - Private Keys}
You can safely send your coins from any exchange to any wallet by withdrawing from one address and sending to another address. For instance, you can own multiple wallets (one on your pc, one on your mobile and a hardware wallet) which provide you with at least the same amount of private keys, public keys and public addresses. You need your private key to sign off on transactions. Hence it would be best if you backed it up somewhere safe. If people somehow obtain your private key, they will be able to access your funds, and your wallet is compromised.


\paragraph{Transact ownership rights}
When you want to send a transaction to another public address, you are transferring ownership rights of assets linked to your wallets public address to that of another public address, which is now only accessible by the corresponding private key of that specific person. Thus, if you somehow lose access to your private key, you will be unable to access your funds. Always make sure to backup your private keys or passwords that you have set up in multiple safe locations.
 
\paragraph{Hashing versus Encryption}
Hashing is great for usage in any instance where you want to compare a value with a stored value, but can't store its plain representation for security reasons. Other use cases could be checking the last few digits of a credit card match up with user input or comparing the hash of a file you have with the hash of it stored in a database to make sure that they\'re both the same.
 
\paragraph{Hashing}
A hash is a string or number generated from a string of text. The resulting string or number is a fixed length and will vary widely with small variations in input. The best hashing algorithms are designed so that it\'s impossible to turn a hash back into its original string.

\paragraph{Encryption}
Encryption turns data into a series of unreadable characters, that aren't of a fixed length. The key difference between encryption and hashing is that encrypted strings can be reversed back into their original decrypted form if you have the right key. Encryption should only ever be used over hashing when it is a necessity to decrypt the resulting message. For example, if you were trying to send secure messages to someone on the other side of the world, you would need to use encryption rather than hashing, as the message is no use to the receiver if they cannot decrypt it. 

\section{Crypto Slang}

Bitcointalk forums, Facebook groups, Reddit, Telegram and many other platforms all provide opportunities for crypto enthusiasts to discuss the latest news events, to exchange investment strategies, and to help each other out with some of the more practical aspects of owning, storing and trading cryptocurrency. If you want to make full use of the online communities resources, a wealth of crypto (and often cryptic!) jargon has now become commonplace. Thus, it makes sense that you get to grips with the meaning of such terms when you encounter them.

Crypto slang is colorful, unusual, occasionally indecipherable, and – to more experienced cryptocurrency enthusiasts and traders – often the sign of a wide-eyed newbie. From Lambos to Shitcoins (and rarely the other way around) we present a glossary of the terminology you’ll encounter in the crypto-world.

\paragraph{Altcoin}
All coins that originated after Bitcoin. Altcoins are the alternative cryptocurrencies launched after the success of Bitcoin. Generally, they project themselves as better substitutes to Bitcoin. The success of Bitcoin as the first peer-to-peer digital currency paved the way for many to follow. Many altcoins are trying to target any perceived limitations that Bitcoin has and come up with newer versions with competitive advantages. As the term 'altcoins' means all cryptocurrencies which are not Bitcoin, there are hundreds of altcoins. 

\paragraph{DYOR}
Do Your Own Research. This stresses that you should always investigate yourself and should not rely on others when making your own decisions. There are a lot of people asking advice and there are also many people shilling coins or projects. Make sure you know what you are getting into.

\paragraph{FOMO}
Fear Of Missing Out. Market panic tends to get the better of people who just entered the market. When the market sentiment is largely positive and prices are on the rise, people get this urge that they are missing out on huge returns and start buying at unrealistic and unsustainable price ranges. When the hype is over and people start to see that this positive surge can't sustain itself, panic enters the market and people start selling quickly. This leads to a lot of people getting burned in the end. Always be watchful when the markets are on a rampage.

\paragraph{FUD}
Fear, Uncertainty & Doubt. People who are spreading FUD are mostly people who are spreading rumors or nonsense based on nothing but their own opinion or experience. Everybody has an opinion and it's very easy to lose heart if you have lost a significant portion of an investment. Media also spreads FUD occasionally.

\paragraph{HODL}
Hold On for Dear Life. Whatever you do, don't sell your coins. Hold them throughout both good and bad times. This term originated in a bitcointalk forum post where somebody misspelled "hold" and coined the term "hodl". It proceeded to becoming legendary.

\paragraph{Moon}
Prices or coins that are going to the moon (when moon) indicate a dramatic surge in prices. The moon bit almost never happens though and a lot of people end up losing money. Be careful when you notice a lot of people drooling over prices.

\paragraph{Lambo}
Lambo stands for Lamborghini, which is a very expensive sports car. "When Lambo" practically means the same as "When Moon" since they both refer to anticipated price surges where people expect to make so much money that they can buy a Lamborghini. It does occasionally happen that someone is able to make huge returns but usually a lot of other people get burned in the process.

\paragraph{Shitcoin}
Shitcoin is a pejorative term used to describe an altcoin that has become worthless. Shitcoin value may disappear because interest failed to materialize, because the altcoin itself was not created in good faith, or because the price was based on speculation.  

\paragraph{BTD or BTFD} 
Buy The (Fucking) Dip. Buying the dips occurs after there has been a significant dip in the price of a security or stock index. Investors practicing this method will increase positions or purchase these newly lower-priced stocks to capitalize on what they hope will be a coming upswing in prices.  

\paragraph{Bag holder}
An informal term used to describe an investor who holds a position in a security which decreases in value until it is worthless. In most cases, the bag holder will hold the position for an extended period during which most of the investment is lost.  

\paragraph{Pump}
A sudden surge in prices of a crypto, stock or other type of security. Could be part of a pump and dump scheme, or based on actual developments or value being created such as new releases, alpha and beta stages etc.

\paragraph{Dump}
A drop in prices which could be caused by a major set back within the project itself, pump and dump schemes or investors selling stocks at high prices in order to take profits.

\paragraph{Pump & Dump}
Pump and dump is a scheme that attempts to boost the price of a stock through recommendations based on false, misleading or greatly exaggerated statements. The perpetrators of this scheme, who already have an established position in the company's stock, sell their positions after the hype has led to a higher share price. This practice is illegal based on securities law and can lead to heavy fines.  

\paragraph{Whale}
A whale is term in the cryptocurrency world used to refer to individuals or entities that hold large amounts of coins or tokens. From the point of view of blockchain and its core decentralized feature, whales cause concern, as the situation could lead to a small number of people having controlling power over the cryptocurrency and if they work together might be able to manipulate prices and markets.  

\paragraph{Bear/Bearish}
A bear is an investor who believes that a particular security or market is headed downward and attempts to profit from a decline in stock prices. Bears are typically pessimistic about the state of a given market. For example, if an investor were bearish on the Standard & Poor's (S&P) 500, that investor would attempt to profit from a decline in the broad market index.  

\paragraph{Bear trap}
A Bear Trap is a technical pattern that occurs when the performance of a stock or an index incorrectly signals a reversal of a rising price trend.  

\paragraph{Bull/Bullish}
A bull is an investor who thinks the market, a specific security or an industry is poised to rise. Investors who adopt a bull approach purchase securities under the assumption that they can sell them later at a higher price. Bulls are optimistic investors who are attempting to profit from the upward movement of stocks.  

\paragraph{Bull trap}
A bull trap is a false signal indicating that a declining trend in a stock or index has reversed and is heading upwards when, in fact, the security will continue to decline. The move "traps" traders or investors that acted on the buy signal and generates losses on resulting long positions. A bull trap may also be referred to as a whipsaw pattern.  

\paragraph{FA}
Fundamental analysis [FA] instead looks at economic and financial factors that influence a business. Fundamental analysis is the cornerstone of investing. In fact, some would say that you aren't really investing if you aren't performing fundamental analysis. Because the subject is so broad, however, it's tough to know where to start. There are an endless number of investment strategies that are very different from each other, yet almost all use the fundamentals. Within the cryptocurrency market, the process differs slightly when compared to investing in regular companies and corporations. Please refer to our own section on performing fundamental analysis for cryptocurrency related projects.

\paragraph{TA}
Technical analysts [TA] typically begin their analysis with charts because technical analysis looks at the price movement of a security and uses this data to predict future price movements. Technical analysts believe that there’s no reason to analyze a company’s financial statements since the stock price already includes all relevant information. Instead, the analyst focuses on analyzing the stock chart itself for hints into where the price may be headed.  

\paragraph{Faucet}
Bitcoin faucets are reward systems that are run on exclusive websites, portals, or digital apps. They are used to reward users by paying them small amounts of satoshi in exchange for completing a task. A bitcoin faucet allows one to earn satoshi, which is a fraction of a bitcoin, either by performing a simple task – like solving a captcha puzzle or completing surveys - or by visiting an advertiser/partner website for a specified period of time. In addition to bitcoin, there are faucets for other cryptocurrencies as well.   