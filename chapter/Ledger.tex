\chapter{HISTORY OF THE LEDGER}
\label{ch:Ledger}

As blockchain technology becomes more and more entrenched in our world and societies, acceptance of its value as a store of value is slowly increasing. News releases regarding Bitcoin (BTC), Ethereum (ETH) and many other cryptocurrencies and blockchain-based projects are becoming more common as startups start to deliver on promises and expectations and more people are gradually becoming aware of the potential of blockchain technology and cryptocurrencies.
 
\begin{quotation}

      \textit{\say{In this exciting time in history, every kind of asset, from tickets to money, and music, can be stored, moved, exchanged, and transacted without an intermediary. People can transact peer-to-peer with the addition of trust by using blockchain technology which offers unique ways of trust through collaboration and cryptography.}}
      \begin{flushright}
        \small{--- \textbf{Tapscott, Don}}
      \end{flushright}
    
\end{quotation}

 
 \section{The Ledger}
The first blockchain component that we're going to discuss is the ledger. However, before we discuss the ledger, let's evaluate the history of the ledger briefly. The story of blockchain is closely linked with the story of accounting. Historically, humans started with no way to prove ownership, and we began with a single entry accounting system.

\subsection{Single-entry accounting}
The single-entry accounting system for the first time in human history allowed us to prove ownership of the asset. The ledger was associated with an owner. The single-entry accounting system has worked for centuries. The issue with single-entry accounting is that it mandated that there was a sole authority, which is the reason why there was the necessity for a king or a queen to control the ledger.\medskip

\subsection{Double entry accounting}
To have international trade, we needed to have at least two authorities. So, for instance, for England to trade with France, we had the owner of the ledger, the single entry ledger, in England, trading with the king or queen of France, who also had their ledger. So we needed a new form of accounting, and that's where double-entry accounting came in. Double-entry accounting is in use up until this moment. Enter blockchain. 

\medskip 




\section{\texorpdfstring{1$^{st}$ generation: triple entry accounting - distributed ledger technology (DLT)}{1st generation: generation: triple entry accounting - distributed ledger technology (DLT)}}
Blockchain is the very first implementation of triple entry accounting, where the network records transactions on the ledger. The third entry and triple entry accounting is cryptography, where we have a cryptographic account of the transaction stored permanently and immutably on the ledger. That is the trust that blockchain can provide.

\medskip 
\tcbset{colback=orange!3!white,fonttitle=\bfseries}
    \begin{tcolorbox}
    [enhanced,
    title=Distributed Ledger Technology (DLT),
    frame style=
    {left color=orange!85!black,right color=yellow!95!black}]
        
           \textit{A ledger is a collection of transactions. It is not a collection of assets. Assets are part of a transaction, but the ledger records the transaction. The network is maintained and run by nodes. Nodes are made up of the users of a network and can be both run and maintained by individuals, companies or institutions.}
       
\end{tcolorbox}
\medskip

The difference with blockchain is that no one owns the ledger, or all of the participants (nodes) own and maintain the ledger. The ledger is distributed (there are nodes everywhere). It is, in other words, decentralised. So, there is a copy of the ledger that exists on every node that exists on the network. So the ledger is a distributed immutable record of a collection of transactions. The Bitcoin was the first asset to be recorded as a transaction on a blockchain ledger.


\section{\texorpdfstring{2$^{nd}$ generation: smart contracts and decentralized applications (DApp)}{2nd generation: smart contracts and decentralized applications (DApp)}}
Blockchain isn't just for transactions. It also extends to contracts. These are called Smart Contracts. So what are they exactly? A smart contract is a digital (coded) software program which self-executes and handles the enforcement, the management, and performance of agreements between parties. Examples of smart contracts include insurance policies, copyrighted content, escrow and lending, wills, and trusts. Smart contracts will revolutionise how we do business as they serve as an intermediary and could potentially eliminate trusted third parties which are required for many businesses today. 
 
As we move to more modern blockchains, we start to look at blockchains such as Ethereum, which not only records the asset on the blockchain,
Ethereum and other public blockchains like Ethereum, they also allow you to have a permanent and immutable collection of code, also known as a smart contract, that runs on the blockchain.
So, the ledger does not store any assets but only records the transactions on the blockchain, and it also holds the code. The code that is stored on the blockchain is a smart contract. Again, a smart contract is a program that runs on the blockchain. The blockchain is a network.

\section{\texorpdfstring{3$^{rd}$ generation: functionality, high performance and design}{3rd generation: functionality, high performance and design}}

Third generation blockchains are considered to be the result of working on more efficient blockchain-like solutions. It may well be that some of these projects are designing post-blockchain technology, which is a new design (and might function completely different) to get rid of the flaws and limitations of the previous generations. Some examples of projects that implement new network designs and architectures are Holochain, Sidechain and Hashgraph. Even though these technologies are different, we can refer to them as the third generation of blockchain because they further develop on some of the blockchain's key characteristics and can be used similarly. Of course, many of the third-gen projects are blockchain-based.\medskip

The main features are wider functionality and better design that helps avoid problems in areas such as:

 \begin{enumerate}[label=(\alph*)]
 \setlength\itemsep{0em}
        \item Speed and scalability (transactions per second, network architecture)
        \item Interoperability (cross-chain, side chains, communication)
        \item Sustainability (energy costs, data usage, efficiency)
        \item Privacy (online identity and sovereignty)
        \item Governance (DAOs, communities, voting)
\end{enumerate}

\section{\texorpdfstring{4$^{th}$ generation: improving technology and driving real-world adoption}{4th generation: improving technology and driving real-world adoption}}
Blockchain has come a long way in the past decade. It has logged some tremendous achievements but has come to some standstill in terms of mass adoption. While big tech-companies and corporations, a few governments, a score of Silicon Valley entrepreneurs and several thousand blockchain startups might be riding the blockchain wave, the technology itself has not yet found its way to local entities such as schools, universities, hospitals, small businesses and ordinary people.

Blockchain 4.0 seeks to break that very divide, by making the technology accessible to institutions and industries, governments and ordinary people.

The blockchain platforms of the future should be easier to use and enable widespread adoption and us-age. The complexity of the underlying blockchain technology is foreseen to evaporate in the background, and it might very well be this moment in time were blockchain technology breaks through. We believe this process of adoption will spark the true disruptive potential the technology holds and propel us into the future. 

There are limitless possibilities with blockchain and other technologies, not just in the now, but with things we haven't even begun to think about yet. Blockchain 4.0 will be the business-friendly, high-performance public chain that can execute real-world use cases and applications.

\section{Emerging markets and early adopters}

When you consider that the cryptocurrency market remains highly volatile and largely unregulated, it is still highly speculative to invest in any of it. The road to mass adoption is not without significant hiccups. We need to give blockchain technology time to mature. If given that time, all these daring and promising startups will have to prove themselves, just like any other company and any other technology. Even though there is much debate around the question how we should deal with blockchain and cryptocurrencies in the coming years, history shows us that, the first reactions to any major scientific breakthroughs were never all-out positive from day one.\medskip

\medskip 
\tcbset{colback=orange!3!white,fonttitle=\bfseries}
    \begin{tcolorbox}
    [enhanced,
    title=The Internet of Value,
    frame style=
    {left color=orange!85!black,right color=yellow!95!black}]
        
           \textit{Create a single open, global payment network, that connects everyone.}
       
\end{tcolorbox}
\medskip

Besides, this market has enormous potential when you think of the possibilities for potential growth on investments in the long term since it is still in its infancy stages. Some of these projects are here to stay, and many more are starting every day, introducing healthy competition that is going to boost the fin-tech industry. There is an ever-increasing demand for skilled people in the fin-tech sector. Businesses cannot seem to find enough software engineers and blockchain developers, and everyone can get into it - take several free courses available on the internet to get you up to speed and develop new skill-sets that might boost your career and give you a new direction - this industry is just getting started. Not to mention the fact that the younger generations have been growing up in the digital age and the impact this makes on their lives cannot be unseen.\medskip




