\chapter{HOW DO CRYPTOCURRENCY WALLETS WORK?}
\label{ch:technical wallets}

Millions of people use cryptocurrency wallets, but there is a considerable misunderstanding about how they work. Unlike traditional \say{pocket} wallets, digital wallets do not store any currency. Cryptocurrencies do not get stored in any single location or exist anywhere in any physical form. All there is are records of transactions stored on the blockchain. Cryptocurrency wallets are software programs that work with public and private key pairs. The wallets provide a user interface via which you can monitor their recorded balance on the network and conduct transactions. When a person sends you Bitcoin or any other type of digital currency, they are permanently signing off ownership of the coins linked to their wallet address, and once the transaction is verified, the network records Bitcoins ownership on your address.

\section{Recovery or seed phrases}
Anytime you set up a wallet; users are provided with a unique recovery seed composed of anywhere from 12 to 24 randomized words (12 and 24 being most common). You are encouraged to write this recovery seed down somewhere safe and never to post it online.
Recovery seeds are considered the most crucial aspect of maintaining the safety of your cryptocurrencies. A recovery seed is your best friend when you lose your paper, hardware, or mobile wallet, as its the only way you can recover your funds and wallet. Many individuals skip writing down their recovery seed code when setting up a wallet because they are too hasty. However, to prevent total financial oblivion, it is crucial to record your recovery seed somewhere safe. If you have not already made a backup of your seed phrase, please do so now before you continue. If you ever lose access to your wallet for whatever reasons, the seed phrase is what you require to recover your lost funds. 


\section{Cryptography}
The field of cryptography is fundamental to many cryptocurrency systems such as Bitcoin. Cryptography is the practice of secure communication in the presence of third parties. In other words, cryptography allows for data to be stored and communicated in such a way that third parties are prevented from reading the contents. Cryptography is utilized in the creation of public and private keys to make cryptocurrency systems a secure network upon which users can safely operate.\medskip

The concept of ownership on a cryptocurrency system is primarily comprised of three interconnected elements:

\begin{enumerate}
\setlength\itemsep{0em}
    \item \textbf{Public and private keys}
    \item \textbf{Public addresses (hashed public key)}
    \item \textbf{Digital signatures} 
\end{enumerate}

\subsection{Public and private keys}
When dealing with cryptocurrency, users are given a public address and a private key to send and receive coins or tokens. The public address is where the funds are deposited and received. However, even though a user has tokens deposited into his address, he won't be able to withdraw them without the unique private key. The public address is generated by hashing the public key. In turn, the public key is created from the private key through a complicated mathematical algorithm (hashing). However, it is near impossible to reverse the process by generating a private key from a public key. This process is illustrated in \cref{flowchart:keys}.\medskip

\tikzset{%
  >={Latex[width=2mm,length=2mm]},
  % Specifications for style of nodes:
            base/.style = {rectangle, rounded corners, draw=black, minimum width=4cm, minimum height=1cm, text centered, font=\sffamily},
            privatekey/.style = {base, fill=blue!30},
            publickey/.style = {base, fill=green!30},
            publicaddress/.style = {base, fill=orange!15, font=\ttfamily},
            }

        


\begin{figure}
  
  \centering
    \begin{tikzpicture}[node distance=2.5cm,every node/.style={fill=white, font=\sffamily}, align=center]
      % Specification of nodes (position, etc.)
      \node (privatekey)       [privatekey]                             {Private Key};
      \node (publickey)        [publickey, below of=privatekey]            {Public Key};
      \node (publicaddress)    [publicaddress, below of=publickey]        {Public Address};
     
      
      \draw[->]             (privatekey) -- node[text width=4cm] {One way hash function(s)} (publickey);
      \draw[->]             (publickey) -- node[text width=4cm] {One way hash function(s)} (publicaddress);
      
      \draw [red, dashed, very thick] (3,-1.25) node {X} (3,-3.75) node {X};
       
      \draw [->,red, dashed, very thick](2,-2.5) .. controls (3,-2.5) and (3,0) .. (2,0);
      \draw [->,red, dashed, very thick](2,-5) .. controls (3,-5) and (3,-2.5) .. (2,-2.5);
        
      \draw [->,green, dashed, very thick](-2,0) .. controls (-3,-0) and (-3,-2.5) .. (-2,-2.5);
      \draw [->,green, dashed, very thick](-2,-2.5) .. controls (-3,-2.5) and (-3,-5) .. (-2,-5);
    \end{tikzpicture}
    
    \vspace{1cm}% NO SPACE!
    
    \begin{tikzpicture}[node distance=2.5cm,every node/.style={fill=white, font=\sffamily}, align=center]
      % Specification of nodes (position, etc.)
      \node (privatekey)       [privatekey]                             {earpiece fiber persecute walmart viable huntsman};
      \node (publickey)        [publickey, below of=privatekey]         {65F0281F30CF3CEC18823EF67DCA8EA8};
      \node (publicaddress)    [publicaddress, below of=publickey]     {8B3A40E6E026B5701932BB4DAAAFFD50CE071D3B0C395859455D856E6F01AD74};
     
      \draw[->]             (privatekey) -- node[text width=4cm] {MD5 Hash} (publickey);
      \draw[->]             (publickey) -- node[text width=4cm] {SHA256} (publicaddress);
    
    \end{tikzpicture}
    \vspace{3mm}
    \caption[Generic representation of cryptographic hash functions]{Generic representation of cryptographic hash function(s). These principles might differ significantly depending on the specific network architecture, design choices and consensus algorithms (if any).}
    \label{flowchart:keys}
    \source{\emph{Online text \& file hashing. \url{http://www.hashemall.com}}}
\end{figure}


    


\section{Public addresses}
Your public address is essentially a secured cryptographic representation of your public key. Think of this address as having a mailbox with a specific string of numbers attached as an identifier and only you can access it with your private key. Some examples of different cryptocurrency public keys or addresses are listed in \cref{tab:publickeys}. Usually, these public addresses will have an automated copy button available and will also be shown as a QR code so you can easily scan it with a mobile device. Remember that for each public address you have both a public and a private key that match it!

\bigskip

\tcbset{colback=orange!3!white,fonttitle=\bfseries}
    \begin{tcolorbox}
    [enhanced,
    title=Public and private key pairs,
    frame style=
    {left color=orange!85!black,right color=yellow!95!black}]

    \say{In simple terms, you can think of the pair just like an email account or a bank login; the address is like your username or email address, and the private key is like your password. However, if you want to send the coins within your wallet to another wallet, you will need your private key. This is exactly like an email account; the email address is a point of reference for users on the email network to send mail to, and the password gives you full access to the privileges of the account-chiefly, the ability to draft and send emails to other accounts.}\cite{Ethos3}

    \end{tcolorbox}

\bigskip


\begin{table}[!htb]
\centering

\caption{Public addresses of a few well-known cryptocurrencies.}
\begin{tabular}{lll} 
\toprule
\textbf{Name} & \textbf{Ticker} & \textbf{Public Address}      \\
                            \midrule
Bitcoin & BTC & 1FER7Ztb6JZoEnomLvacnqkfD1LHzMyi2G              \\
Ethereum & ETH & 0xF8D519c05E11eFa9875Be9eF5DE4AB941de23350                \\
Ripple & XRP & rogue5HnPRSszD9CWGSUz8UGHMVwSSKF6 \\

\bottomrule
\end{tabular}
\label{tab:publickeys}
\end{table}

\subsection{Digital signatures}
Digital signatures play an important role in cryptocurrency systems because they prove ownership of funds and allow the individual in control of those funds to spend them. With Bitcoin, a digital signature is effectively intended to serve three distinct purposes:

\begin{itemize}
\setlength\itemsep{0em}
    \item A digital signature serves as proof that the owner of a private key, who will by extension have ownership of his/her funds, has indeed authorized that those funds can be spent.
    \item A digital signature serves as proof that the authorization is undeniable.
    \item A digital signature proves that the transaction that has been authorized by the signature has not or cannot be modified by anyone after it has been signed.
\end{itemize}

\section{Transactions}
You can safely send your coins from any exchange to any wallet by withdrawing from one address and sending to another address. For instance, you can own multiple wallets (one on your pc, one on your mobile and a hardware wallet) which provide you with at least the same amount of private keys, public keys and public addresses. You need your private key to sign off on transactions. Hence it would be best if you backed it up somewhere safe. If people somehow obtain your private key, they will be able to access your funds, and your wallet is compromised.\medskip

When you want to send a transaction to another public address, you are transferring ownership rights of assets linked to your wallets public address to that of another public address, which is now only accessible by the corresponding private key of that specific person. Thus, if you somehow lose access to your private key, you will be unable to access your funds. Always make sure to back up your private keys or passwords that you have set up in multiple safe locations.

\section{Hashing versus encryption}
Hashing is great for usage in any instance where you want to compare a value with a stored value, but can't store its plain representation for security reasons. Other use cases could be checking the last few digits of a credit card match up with user input or comparing the hash of a file you have with the hash of it stored in a database to make sure that they're both the same.

\medskip
\tcbset{colback=orange!3!white,fonttitle=\bfseries}
    \begin{tcolorbox}
    [enhanced,
    title=Hashing,
    frame style=
    {left color=orange!85!black,right color=yellow!95!black}]

    A hash is a string or number generated from a string of text. The resulting string or number is a fixed length and will vary widely with small variations in input. The best hashing algorithms are designed so that it's impossible to turn a hash back into its original string.

\tcblower

\footnotesize \begin{enumerate}
                    \item MD5 - MD5 is the most widely known hashing function. It produces a 16-byte hash value, usually expressed as a 32 digit hexadecimal number. Recently a few vulnerabilities have been discovered in MD5, and rainbow tables have been published which allow people to reverse MD5 hashes made without good salts.
                    \item SHA - There are three different SHA algorithms -- SHA-0, SHA-1, and SHA-2. SHA-0 is very rarely used, as it has contained an error which was fixed with SHA-1. SHA-1 is the most commonly used SHA algorithm and produces a 20-byte hash value.
                    \item SHA-2 consists of a set of 6 hashing algorithms and is considered the strongest. SHA-256 or above is recommended for situations where security is vital. SHA-256 produces 32-byte hash values.
                \end{enumerate}
\end{tcolorbox}
\medskip
    
Encryption should only ever be used over hashing when it is a necessity to decrypt the resulting message. For example, if you were trying to send secure messages to someone on the other side of the world, you would need to use encryption rather than hashing, as the message is no use to the receiver if they cannot decrypt it.

\medskip
\tcbset{colback=orange!3!white,fonttitle=\bfseries}
    \begin{tcolorbox}
    [enhanced,
    title=Encryption,
    frame style=
    {left color=orange!85!black,right color=yellow!95!black}]

    Encryption turns data into a series of unreadable characters, that aren't of a fixed length. The key difference between encryption and hashing is that encrypted strings can be reversed back into their original decrypted form if you have the right key.

\tcblower

\footnotesize \begin{enumerate}
                    \item AES - AES is the "gold standard" when it comes to symmetric key encryption, and is recommended for most use cases, with a key size of 256 bits. Learn more about AES.

                    \item PGP - PGP is the most popular public-key encryption algorithm. Learn more about PGP.
                    
                \end{enumerate}

\end{tcolorbox}
\medskip

